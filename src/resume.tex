\documentclass[10pt,a4paper]{article}
\usepackage{fontspec}
\usepackage{geometry}
\usepackage{enumitem}
\usepackage{hyperref}
\usepackage{titlesec}
\usepackage{setspace}
\usepackage{multicol}

% Шрифт с поддержкой кириллицы
\setmainfont{DejaVu Sans}  % или Liberation Sans, Nimbus Sans

% Поля
\geometry{left=1.2cm, right=1.2cm, top=1.5cm, bottom=1.5cm}

% Стили заголовков
\titleformat{\section}{\large\bfseries\uppercase}{}{0em}{}[\titlerule]
\titlespacing*{\section}{0pt}{12pt}{6pt}

% Гиперссылки
\hypersetup{
    colorlinks=true,
    urlcolor=blue,
    linkcolor=black
}

% Отступы списков
\setlist[itemize]{noitemsep, topsep=0pt}
\setlength{\parindent}{0pt}
\pagestyle{empty}

\begin{document}

\begin{center}
    {\LARGE \textbf{Брагин Дмитрий}} \\[4pt]
    \small
    \textbar{} 
    @DmitiyBragin \textbar{} 
    \href{mailto:dimanb1982@gmail.com}{dimanb1982@gmail.com} \textbar{} 
    \href{https://github.com/bragind}{github.com/bragind} \\
    Гражданство: Россия \textbar{} Удалённая работа
\end{center}

\vspace{8pt}

\section*{Цель}
Инженер с 16+ годами опыта в IT, в настоящее время специализируюсь на Python, C/C++, Linux, автоматизации и ML. Ищу позицию, Systems Engineer / AI Integrator

\section*{Опыт работы}
\textbf{Ведущий инженер} \hfill Окт 2024 – н.в. \\
ООО «Интегрированные системы безопасности»
\begin{itemize}
    \item Интеграция электронных модулей и встраиваемых блоков в комплексные технические решения.
    \item Отладка аппаратно-программных интерфейсов.
    \item Автоматизация сбора телеметрии и тестирования.
    \item Ведение технической документации и взаимодействие с командами разработки ПО и конструкторского отдела.
\end{itemize}

\textbf{Старший инженер} \hfill Окт 2011 – Окт 2024 \\
Альфа-Банк (Россия), Москва
\begin{itemize}
    \item Обеспечение отказоустойчивости ИТ-инфраструктуры (серверы, сети, мониторинг).
    \item Диагностика и устранение критических инцидентов.
    \item Взаимодействие с внутренними DevOps-командами и вендорами.
\end{itemize}

\textbf{Администратор вычислительной сети} \hfill Ноя 2009 – Окт 2011 \\
МФЦ Пензенской области
\begin{itemize}
    \item Администрирование корпоративной сети (VLAN, маршрутизация, 50+ рабочих мест).
    \item Настройка сетевого оборудования и серверов Windows Server.
\end{itemize}

\section*{Образование}
\textbf{НИЯУ «МИФИ»}, Москва \hfill 2024–2026 \\
Магистратура, Прикладная математика и информатика \\
Специализация: \textit{Машинное обучение}

\textbf{Курсы повышения квалификации (МИФИ, 2025)}:
\begin{itemize}
    \item Администрирование Linux
    \item Программирование на C/C++ в UNIX-подобных системах
\end{itemize}

\textbf{ПГПУ им. В.Г. Белинского}, Пенза \hfill 2005 \\
Физико-математический факультет, Физика

\section*{Навыки}
\begin{multicols}{2}
\begin{itemize}
    \item \textbf{Языки}: Python, C/C++, Bash
    \item \textbf{Системы}: Linux, Windows
    \item \textbf{Инструменты}: Docker, Git, Ansible, Terraform, Kubernetes, PostgreSQL,Prometheus, Grafana, MLflow, DVC
    \item \textbf{ML/DL}: Pandas, NumPy, scikit-learn, PyTorch
    \item \textbf{Практики}: автоматизация, документирование, тестирование, CI/CD
\end{itemize}
\end{multicols}

\section*{Академические проекты МИФИ с упором на развертывание в продакшене}
\begin{itemize}
    \item \textbf{Анализ COVID-19 по рентгеновским снимкам} — классификация изображений (PyTorch, OpenCV)
    \item \textbf{Прогнозирование цен на акции} — анализ временных рядов (scikit-learn, PyTorch, интеграция с Telegram)
    \item \textbf{Определение планктона} — CV-классификация планктона
    \item \textbf{Предсказание дефолта клиента (MLOps-пайплайн)} — FastAPI, Docker, DVC, GitHub Actions,Terraform,Kubernetes, MLFlow
    \item \textbf{RAG-платформа для семантического поиска технических статей} — 
      архитектура интеграции разнородных компонентов для интеллектуального поиска:
      \begin{itemize}
          \item \textbf{Системная интеграция}: оркестрация взаимодействия между 
                парсерами (Habr), векторной БД (ChromaDB), LLM (Ollama) и UI 
                (Streamlit) через единый RAG Orchestrator
          \item \textbf{Безопасность и изоляция}: ролевая модель доступа (гость/пользователь/админ), 
                аутентификация bcrypt + PyJWT, изоляция пользовательских коллекций 
                на уровне данных
          \item \textbf{Надёжность пайплайна обработки}: валидация входных документов 
                (PDF/HTML), обработка ошибок парсинга, чанкинг с контролем перекрытия 
                для сохранения семантической целостности
          \item \textbf{Модульность}: чёткое разделение ответственности 
                (auth/ingest/core/generation/ui) — готовность к замене компонентов 
                (например, переход с ChromaDB на pgvector без изменения бизнес-логики)
          \item \textbf{Конфигурируемость}: параметризация эмбеддингов, размера чанков, 
                модели LLM — поддержка A/B-тестирования и постепенного развёртывания
      \end{itemize}
      
      \textit{Проект разработан в рамках хакатона Cloud.ru как архитектурно зрелый MVP, 
      демонстрирующий системное проектирование интеграционных решений.}
    \item Все проекты: \url{https://github.com/bragind}
\end{itemize}

\section*{Дополнительно}
Английский язык — чтение технической документации и спецификаций, активно повышаю уровень до B2+.

Интересы: архитектура систем с машинным обучением на стыке software/hardware, 
робототехнические платформы, автоматизация жизненного цикла разработки.

\end{document}
