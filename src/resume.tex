\documentclass[10pt,a4paper]{article}
\usepackage{fontspec}
\usepackage{geometry}
\usepackage{enumitem}
\usepackage{hyperref}
\usepackage{titlesec}
\usepackage{setspace}
\usepackage{multicol}

% Шрифт с поддержкой кириллицы
\setmainfont{DejaVu Sans}  % или Liberation Sans, Nimbus Sans

% Поля
\geometry{left=1.2cm, right=1.2cm, top=1.5cm, bottom=1.5cm}

% Стили заголовков
\titleformat{\section}{\large\bfseries\uppercase}{}{0em}{}[\titlerule]
\titlespacing*{\section}{0pt}{12pt}{6pt}

% Гиперссылки
\hypersetup{
    colorlinks=true,
    urlcolor=blue,
    linkcolor=black
}

% Отступы списков
\setlist[itemize]{noitemsep, topsep=0pt}
\setlength{\parindent}{0pt}
\pagestyle{empty}

\begin{document}

\begin{center}
    {\LARGE \textbf{Брагин Дмитрий}} \\[4pt]
    \small
    Пенза \textbar{} 
    +7 (967) 701-84-12 \textbar{} 
    \href{mailto:dimanb1982@gmail.com}{dimanb1982@gmail.com} \textbar{} 
    \href{https://github.com/bragind}{github.com/bragind} \\
    Гражданство: Россия \textbar{} Удалённая работа
\end{center}

\vspace{8pt}

\section*{Цель}
Инженер с 16+ годами опыта в IT, в настоящее время специализируюсь на Python, C/C++, Linux, автоматизации и ML. Ищу позицию Backend Developer/ Software Engineer / System Engineer / ML/AI Developer

\section*{Опыт работы}
\textbf{Ведущий инженер} \hfill Окт 2024 – н.в. \\
ООО «Интегрированные системы безопасности»
\begin{itemize}
    \item Интеграция электронных модулей и встраиваемых блоков в комплексные технические решения.
    \item Отладка аппаратно-программных интерфейсов: RS-485, Ethernet.
    \item Автоматизация сбора телеметрии и тестирования с помощью \textbf{Python}, \textbf{Bash} и \textbf{C/C++}.
    \item Ведение технической документации и взаимодействие с командами разработки ПО и конструкторского отдела.
\end{itemize}

\textbf{Старший инженер} \hfill Окт 2011 – Окт 2024 \\
Альфа-Банк (Россия), Москва
\begin{itemize}
    \item Обеспечение отказоустойчивости ИТ-инфраструктуры (серверы, сети, мониторинг).
    \item Диагностика и устранение критических инцидентов.
    \item Автоматизация операций мониторинга и отчётности на \textbf{Bash/Python}.
    \item Взаимодействие с внутренними DevOps-командами и вендорами.
\end{itemize}

\textbf{Администратор вычислительной сети} \hfill Ноя 2009 – Окт 2011 \\
МФЦ Пензенской области
\begin{itemize}
    \item Администрирование корпоративной сети (VLAN, маршрутизация, 50+ рабочих мест).
    \item Настройка сетевого оборудования и серверов Windows Server.
\end{itemize}

\section*{Образование}
\textbf{НИЯУ «МИФИ»}, Москва \hfill 2024–2026 \\
Магистратура, Прикладная математика и информатика \\
Специализация: \textit{Машинное обучение}

\textbf{Курсы повышения квалификации (МИФИ, 2025)}:
\begin{itemize}
    \item Администрирование Linux
    \item Программирование на C/C++ в UNIX-подобных системах
\end{itemize}

\textbf{ПГПУ им. В.Г. Белинского}, Пенза \hfill 2005 \\
Физико-математический факультет, Физика

\section*{Навыки}
\begin{multicols}{2}
\begin{itemize}
    \item \textbf{Языки}: Python, C/C++, Bash
    \item \textbf{Системы}: Linux, Docker, Git, CI/CD
    \item \textbf{DevOps}: Ansible, PostgreSQL, Grafana, MLflow, DVC
    \item \textbf{ML/DL}: Pandas, NumPy, scikit-learn, PyTorch
    \item \textbf{Практики}: автоматизация, документирование, тестирование
\end{itemize}
\end{multicols}

\section*{Проекты (open source)}
\begin{itemize}
    \item \textbf{Анализ COVID-19 по рентгеновским снимкам} — классификация изображений (PyTorch, OpenCV)
    \item \textbf{Прогнозирование цен на акции} — временные ряды (scikit-learn, PyTorch)
    \item \textbf{Определение планктона} — CV-классификация подводных изображений
    \item \textbf{MLOps-пайплайн} — FastAPI, Docker, DVC, GitHub Actions
    \item Все проекты: \url{https://github.com/bragind}
\end{itemize}

\section*{Дополнительно}
Английский — A1 (техническая документация читаю свободно).  
Интересы: робототехника, автономные системы, автоматизация.

\end{document}